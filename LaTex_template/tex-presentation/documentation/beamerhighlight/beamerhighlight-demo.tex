\documentclass[10pt]{beamer}
%!TEX encoding = UTF-8 Unicode
%!TEX TS-program = pdflatex
%!TEX spellcheck = English
\usepackage{etex}
\usepackage[utf8]{inputenc}
\usepackage[TS1,T1]{fontenc}
\usepackage{xspace}
\usepackage{amsmath,amssymb}
\usepackage[english]{babel}
%\usepackage{tikz}

\usepackage{lmodern}

\usepackage{beamerhighlight}


%% libraries for snake paths
\usetikzlibrary{decorations}
\usetikzlibrary{decorations.pathmorphing}

%% library for star node
\usetikzlibrary{shapes.geometric}

%% library for computation of coordinates
\usetikzlibrary{calc}


\begin{document}

%%%%%%%%%%%%%%%%%%%%%%%%%%%%%%%%%%%%%%%%%%%%%%%%%%%%%%%%%%%%%%%
\begin{frame}{Comment of math terms, 1}
\setcounter{beamerpauses}{2}

\uncover<+->{Suspendisse potenti.}

\begin{equation*}
%% default styles
\commentmathwithtext<+->[%position={(60:2cm)},
		append text node style={},
		append math node style={},
		append link style={},
		append to style={}
		]{\int_a^b f(x) dx}{This is an integral} 
+ B = C + 
%% change relative position
%% change the style of the text/comment node
\commentmathwithtext<+>[position={(30:3cm)},
		append text node style={align=left,text width=10em},
		append math node style={},
		append link style={},
		append to style={}
		]{\sum_{n=0}^{\infty} a_n}{This is a sum on $a_n$, from $n=0$ to $n=\infty$}
+ \sum_{n=0}^{\infty} b_n
\end{equation*}

\begin{equation*}
%% change relative position
%% change the style of the text/comment node
%% change the style of the math node
%% change the style of the background node
%% change the style of the link and the "to"
\commentmathwithtext<+>[position={(0:4.5cm)},
		append text node style={text=structure,draw=structure,fill=structure!10,line width=1pt,outer sep=3pt},
		append math node style={outer sep=2pt,draw=structure,fill=structure!10,line width=1pt},
		append link style={structure,line width=2pt,->},
		append to style={bend right=18},
		math color=structure
		]{\int_a^b f(x) dx}{\bfseries This is the same integral}  = 0
\end{equation*}

{%% begin of a group where the colors are changed
\colorlet{mathnodecolor}{magenta!60}
\begin{equation*}
A + A + 
%% change relative position
%% change the style of the text/comment node
%% change the style of the math node
%% change the style of the background node
%% change the style of the link
\commentmathwithtext<+>[position={(-3cm,1.1cm)},
				append text node style={fill=mathnodecolor!50,inner sep=5pt},
				append math node style={fill=none,draw=mathnodecolor,line width=1pt},
				append link style={decorate,decoration={snake,amplitude=1pt}},
				append to style={},
				math color=mathnodecolor
				]{A}{This is just $A$} + A + A = 0
\end{equation*}
}% end of the group for change of colors

Praesent ante turpis, ultrices condimentum fringilla sed. 

Vestibulum ante ipsum primis in faucibus orci luctus. 

Phasellus felis augue, consequat volutpat bibendum id, ultrices a metus.

\uncover<+->{Donec nec ipsum et ipsum pellentesque dictum in vel turpis. }



\end{frame}


%%%%%%%%%%%%%%%%%%%%%%%%%%%%%%%%%%%%%%%%%%%%%%%%%%%%%%%%%%%%%%%
\begin{frame}{Comment of math terms, 2}

\uncover<+->{A test with a star}

{%% beginning of a group to change the style
\colorlet{mathnodecolor}{violet!40}
\tikzset{append text node style={inner sep=4pt,circle}}
\tikzset{append math node style={inner sep=0.5pt,circle}}
\tikzset{append link style={line width=1pt}}

\begin{equation*}
\begin{pmatrix}
\commentmathwithtext<+->[position={(135:2cm)}]{1}{Un $1$}
 & 
\commentmathwithtext<+->[position={(90:2cm)}]{2}{Un $2$}
 & 
\commentmathwithtext<+->[position={(45:2cm)}]{3}{Un $3$}
 \\
\commentmathwithtext<+->[position={(180:2cm)}]{4}{Un $4$}
 &
{\colorlet{mathnodecolor}{orange!60}
%% change only this one
\commentmathwithtext<+->[position={(58:4cm)},
				append text node style={inner sep=0pt,shape=star, star points=6,star point ratio=2}
				]{5}{Un $5$}
}
 &
\commentmathwithtext<+->[position={(0:2cm)}]{6}{Un $6$}
 \\
\commentmathwithtext<+->[position={(225:2cm)}]{7}{Un $7$}
 &
\commentmathwithtext<+->[position={(270:2cm)}]{8}{Un $8$}
 &
\commentmathwithtext<+->[position={(315:2cm)}]{9}{Un $9$}
\end{pmatrix}
\end{equation*}
}%% end of the group for change of style

\end{frame}


%%%%%%%%%%%%%%%%%%%%%%%%%%%%%%%%%%%%%%%%%%%%%%%%%%%%%%%%%%%%%%%
\begin{frame}{Comment of math terms, 3}

\uncover<+->{A big formula where some terms are commented.}

{%% beginning of a group to change the style
\colorlet{mathnodecolor}{orange!60}
\tikzset{math color=orange}
\tikzset{append text node style={outer sep=2pt,text=structure,font=\bfseries\mathversion{bold}}}
\tikzset{append math node style={inner sep=1pt,outer sep=2pt,fill=none}}
\tikzset{append link style={>=latex,->,orange!80,line width=1.5pt}}

\uncover<+->{\begin{multline*}
H = \int_{-\infty}^L \frac{dx}{2\pi} \left[ 
\frac{v_F}{%
\commentmathwithtext<+->[position={(250:2cm)},
		append text node style={align=left,text width=13.2em},
			]{K^2}{Luttinger parameter\\ ($K=1$ without interactions)}
} \left(\frac{\partial \phi}{\partial x} \right)^2 + v_F \left(\frac{\partial \theta}{\partial x} \right)^2 \right]
- 
\commentmathwithtext<+->[position={(40:2cm)}]{V}{Backscattering}
\cos(2 \phi(x=0)) \\
+ \frac{1}{\pi^2} E_C \left[ \phi(x=0) - \left( \frac{\pi C 
\commentmathwithtext<+->[position={(-60:2cm)}]{V_g}{Gate voltage}
}{\lvert e \rvert} + k_F L \right) \right]^2
\end{multline*}
}% uncover
}%% end of the group for change of style


\end{frame}


%%%%%%%%%%%%%%%%%%%%%%%%%%%%%%%%%%%%%%%%%%%%%%%%%%%%%%%%%%%%%%%
\begin{frame}{Fleeting box}
%\framesubtitle{}
%\setcounter{beamerpauses}{2}

\uncover<+->{Text before}

\begin{itemize}
\item<+-> First item
%% default style of the fleeting box
\fleetingbox<.>{Lorem ipsum dolor sit amet, consectetur adipiscing elit. Integer malesuada, odio et bibendum adipiscing, diam mi suscipit arcu, venenatis ullamcorper libero nulla non purus. Aliquam nulla justo, iaculis a vestibulum a, iaculis tristique nisl. Aenean tempus eros id purus convallis ac ornare elit tempor.}

\item<+-> Second item
%% change style of the appearance
{\tikzset{append fleeting box style={top color=structure!5,bottom color=structure!20,xshift=15pt,yshift=10ex,text width=0.7\linewidth}}
%\tikzset{append fleeting box style={top color=structure!5,bottom color=structure!20,anchor=south east,xshift=0pt,yshift=0pt,text width=0.7\linewidth,rotate=30}}
\fleetingbox<.>{Lorem ipsum dolor sit amet, consectetur adipiscing elit. Integer malesuada, odio et bibendum adipiscing, diam mi suscipit arcu, venenatis ullamcorper libero nulla non purus. Aliquam nulla justo, iaculis a vestibulum a, iaculis tristique nisl. Aenean tempus eros id purus convallis ac ornare elit tempor.}

}% local redefinition of style

\item<+-> Third item
%% change style in the optionnal argument
\fleetingbox<.>[draw=red,line width=2pt,text width=0.8\linewidth,fill=none,font=\small\slshape]{Lorem ipsum dolor sit amet, consectetur adipiscing elit. Integer malesuada, odio et bibendum adipiscing, diam mi suscipit arcu, venenatis ullamcorper libero nulla non purus. Aliquam nulla justo, iaculis a vestibulum a, iaculis tristique nisl. Aenean tempus eros id purus convallis ac ornare elit tempor.}
\end{itemize}

\uncover<+->{Text after}

\end{frame}


%%%%%%%%%%%%%%%%%%%%%%%%%%%%%%%%%%%%%%%%%%%%%%%%%%%%%%%%%%%%%%%
\begin{frame}{Highlight node}
%\framesubtitle{}
%\setcounter{beamerpauses}{2}


\begin{itemize}
\item<+-> Lorem ipsum dolor sit amet, consectetur adipiscing elit.
\begin{align*}
\inserthightlightnode{node integral left}{\vphantom{$\displaystyle \int_a^b$}} 
\int_a^b f(x) dx 
\inserthightlightnode{node integral right}{\vphantom{$\displaystyle \int_a^b$}}
&= 10 \\[4pt]
\inserthightlightnode{node sum left}{\vphantom{$\displaystyle \sum_{n=0}^{\infty}$}} 
\sum_{n=0}^{\infty} a_n 
\inserthightlightnode{node sum right}{\vphantom{$\displaystyle \sum_{n=0}^{\infty}$}}
&= 10
\end{align*}

\item<+-> Integer malesuada, odio et bibendum adipiscing, diam mi suscipit arcu, venenatis ullamcorper libero nulla non purus.

\item<+->  Aliquam nulla justo, iaculis a vestibulum a, iaculis tristique nisl. 

\end{itemize}

\hightlightnode<+-.(2)>{(node integral left) (node integral right)}%
\uncover<+->{Aenean tempus eros id purus convallis ac ornare elit tempor.}%
\hightlightnode<+-.(1)>{(node sum left) (node sum right)}%
{\tikzset{append highlight node style={inner sep=1pt,line width=1pt,draw=blue,fill=blue!20,opacity=0.3}}%
\hightlightnode<+>{(node integral left) (node sum right)}}%

\uncover<+->{Aenean tempus eros id purus convallis ac ornare elit tempor.}%
%% possible to create links between nodes
\uncover<.->{\tikz[remember picture,overlay]{%
\node[highlight node style,align=left,fit=(node integral left) (node integral right)] (node integral) {};%
\node[highlight node style,align=left,fit=(node sum left) (node sum right)] (node sum) {};%
\draw[red,line width=1.5pt,<->,rounded corners=3pt] (node sum) -- ($(node sum)+(-2cm,0pt)$) |- (node integral);
}}

\end{frame}





\end{document}

